% HMC Math dept HW template example
% v0.04 by Eric J. Malm, 10 Mar 2005
\documentclass[12pt,letterpaper,boxed]{hmcpset}
\usepackage[francais]{babel}
\usepackage[utf8x]{inputenc}
\usepackage[T1]{fontenc}
\usepackage{enumitem}
% set 1-inch margins in the document
\usepackage[margin=1in]{geometry}

\usepackage{empheq}




\newcommand*\widefbox[1]{\fbox{\hspace{2em}#1\hspace{2em}}}
% include this if you want to import graphics files with /includegraphics
\usepackage{graphicx}

\newcommand{\property}{\mathbf{P}}


% info for header block in upper right hand corner
\name{Tristan Stérin}
\class{FDI}
\assignment{DM \#2}
\duedate{21/12/2015}

\begin{document}

\problemlist{Exercice 1}

\begin{problem}[Question 1]
Montrer que les fonctions suivantes sont primitves récursives.
\begin{itemize}
  \item[(\textit{a})] Les fonctions constantes $c_{a}(n)=a$.
  \item[(\textit{b})] La fonction \textit{somme} définie par $somme(n,m) = n+m$.
  \item[(\textit{c})] La fonction \textit{prédécesseur} p définie par $p(n) = max(0,n-1)$.
  \item[(\textit{d})] La fonction \textit{prod} définie par $prod(n,m) = nm$.
  \item[(\textit{e})] La fonction \textit{eq0} définie par $eq0(m) = 1$ si $m = 0$ et $eq0(m) = 0$ sinon.
  \item[(\textit{f})]  La fonction \textit{eq} définie par $eq(n,m) = 1$ si $m = n$ et $eq(n,m) = 0$ sinon.
  \item[(\textit{g})] Toute fonction $f : N \to \mathbb{N}$ à support fini : $$\exists E \subseteq \mathbb{N}, \, |E| < \infty \quad x \notin E \implies f(x) = 0$$.
\end{itemize}


\end{problem}

\begin{solution}[(a)]
Soit $a \in \mathbb{N}$.\\ On montre que : 
$$  \boxed{\forall n \in \mathbb{N}, \, c_{a}(n) = s^{a}(\odot(n))} $$ 
\\

\noindent Avec $s^{a}$ $s$ composé $a$ fois. \\
Une récurrence sur $a$ montre que $s^a(0) = a$. \\
Écrit sous cette forme on constate que $c_{a}$ est récursive primitive par $a+1$ applications du schéma de composition. \\


%\begin{align*}
 %\limsup_{n \to \infty} \, (a_n + b_n)
 % &= \lim_{n \to \infty} C_n \\
  %&\leq \lim_{n \to \infty} \, (A_n + B_n)
  %= \lim_{n \to \infty} A_n + \lim_{n \to \infty} B_n
  %= \limsup_{n \to \infty} a_n + \limsup_{n \to \infty} b_n.
%\end{align*}
\end{solution}

\begin{solution}[(b)]
On définit \textit{somme'} : 

\ \\
\begin{empheq}[box=\widefbox]{align*}
  \forall (n,m) \in \mathbb{N}^{2}, \quad & somme'(n,0) = n \\ 
  & somme'(n,m+1) = (s \circ p^{3}_{3})(n,m,somme'(n,m))
\end{empheq}
\ \\

\noindent$somme'$ est PR par application une fois du schéma de récurrence primitive : $s \circ p^{3}_{3}$ est PR par composition. 
\newpage
\noindent Soit $n \in \mathbb{N}$, on montre par récurrence sur $m\in\mathbb{N}$ :
$$ \property(m) \, : \, \text{"}somme'(n,m) = somme(n,m) = n+m\text{"}$$

\begin{itemize}
\item 
$\property(0)$ : $somme'(n,0) = n = n + 0$

\item \textbf{Hérédité} : 
Soit $m \in \mathbb{N}$ tel que $\property(m)$, on a :
\begin{align*} somme'(n,m+1) & = (s \circ p^{3}_{3})(n,m,somme'(n,m)) \\
					      & = s(somme'(n,m)) \\
					      & = s(n+m) \\
					      & = (n + m) + 1 \\
					      & = n + (m+1) \\
					     \end{align*}
D'où $\property(m+1)$. 
\end{itemize}
\ \\
Donc $somme = somme'$ et $somme$ est PR.



%\begin{align*}
 %\limsup_{n \to \infty} \, (a_n + b_n)
 % &= \lim_{n \to \infty} C_n \\
  %&\leq \lim_{n \to \infty} \, (A_n + B_n)
  %= \lim_{n \to \infty} A_n + \lim_{n \to \infty} B_n
  %= \limsup_{n \to \infty} a_n + \limsup_{n \to \infty} b_n.
%\end{align*}
\end{solution}

\begin{solution}[(c)]
L'énoncé est un peu ambigü comme soulevé dans la FAQ. \\
On considère malgré les définitions que les $\mathbb{N}^0 \to \mathbb{N}$ sont PR 
(les constantes sont PR, ça justifie). \\
Ainsi on peut définir des fonctions à une seule variable en schéma de récursion primitive. \\
On montre que p vérifie :
\ \\
\begin{empheq}[box=\widefbox]{align*}
    & p(0) = 0 \\ 
  \forall n \in \mathbb{N}, \quad & p(n+1) =  p^{2}_{1}(n,p(n))
\end{empheq}
\ \\

En effet, $p(0) = max(0,-1) = 0$ et pour $n \in \mathbb{N}$ : $$p(n+1) = max(0,n) = n = p^{2}_{1}(n,p(n))$$
Sous cette forme on constate que $p$ est PR.


\end{solution}


\begin{solution}[(d)]

On définit \textit{prod'} : 

\ \\
\begin{empheq}[box=\widefbox]{align*}
  \forall (n,m) \in \mathbb{N}^{2}, \quad & prod'(n,0) = 0 \\ 
  & prod'(n,m+1) = somme_{13}(n,m,prod'(n,m)) \\
  \text{avec : }&somme_{13}(x_{1},x_{2},x_{3}) = somme(p^{3}_{1}(x_{1},x_{2},x_{3}),p^{3}_{3}(x_{1},x_{2},x_{3}))
\end{empheq}
\ \\

\noindent$somme_{13}$ est PR par schéma de composition et 1b) donc $prod'$ est PR par schéma de récurrence.
\newpage
 \noindent Soit $n \in \mathbb{N}$, on montre par récurrence sur $m\in\mathbb{N}$ :
$$ \property(m) \, : \, \text{"}prod'(n,m) = prod(n,m) = n*m\text{"}$$

\begin{itemize}
\item 
$\property(0)$ : $prod'(n,0) = 0 = n*0$

\item \textbf{Hérédité} : 
Soit $m \in \mathbb{N}$ tel que $\property(m)$, on a :
\begin{align*} prod'(n,m+1) & = somme_{13}(n,m,prod'(n,m)) \\
					      & = n+prod'(n,m) \\
					      & = n+n*m \\
					      & = n*(m+1) \\
					     \end{align*}
D'où $\property(m+1)$. 
\end{itemize}
\ \\
Donc $prod = prod'$ et $prod$ est PR.
\end{solution}


\begin{solution}[(e)]
On montre que $eq0$ vérifie :
\ \\
\begin{empheq}[box=\widefbox]{align*}
  & eq0(0) = 1 \\ 
  \forall n \in \mathbb{N}, \quad & eq0(n+1) =  (\odot \circ p^{2}_{1})(n,eq0(n))
\end{empheq}
\ \\

En effet, $eq(0) = 1$ et pour $n \in \mathbb{N}, \, \, n+1 \neq 0$ : $$eq0(n+1) = 0 = \odot(n)$$
Sous cette forme on constate que $eq0$ est PR.
\end{solution}

\begin{solution}[(f)]
On montre que $eq$ vérifie :
\ \\
\begin{empheq}[box=\widefbox]{align*}
  \forall (n,m) \in \mathbb{N}^2, \quad & eq(n,m) =  (eq0 \circ somme)(sub(n,m),sub(m,n))
\end{empheq}
Avec : 
\begin{empheq}[box=\widefbox]{align*}
 & sub(n,0) = n \\
  \forall (n,m) \in \mathbb{N}^2, \quad & sub(n,m+1) =  (p \circ p^{3}_{3})(n,m,sub(n,m))
\end{empheq}
\\
On constate que $sub$ est PR. 
\\ Une récurrence montre que $sub(n,m) = max(0,n-m)$. \\
Soient $(n,m) \in \mathbb{N}^2$ : 
\begin{itemize}
\item Si $n = m$,  $\, \, sub(n,m) = sub(m,n) = 0$ et donc : \\ $eq(n,m) = 1 = eq0(sub(n,m)+sub(m,n))$
\item Si $n \neq m$, mettons $n > m$. \\ Alors $sub(n,m) = n - m \neq 0$ et $sub(m,n) = 0$ aisni $eq0(sub(n,m)+sub(m,n)) = 0$ \\
Ainsi $eq(n,m) = 0 = eq0(sub(n,m)+sub(m,n))$
\end{itemize}
\ \\
Mis sous cette forme, on constate que $eq$ est PR.

\end{solution}

\begin{solution}[(g)]
Soit $f : \mathbb{N} \to \mathbb{N}$ à support fini. \\
Il existe $k \in \mathbb{N}$ et  $E \subseteq \mathbb{N} = \{x_{1}, \, \dots \, , \, x_{k} \}$, $|E|=k$, tels que :
$$ \forall x \in \mathbb{N} \quad x \notin E \implies f(x) = 0$$
Si $k = 0$, $f=\odot$ et le résultat est connu. \\
On suppose donc $k \geq 1$. \\
On note pour $1 \leq i \leq k \quad y_{i} = f(x_{i})$. \\
On définit $f'$ : 
\ \\
\begin{empheq}[box=\widefbox]{align*}
  \forall x \in \mathbb{N}, \quad & f'(x) =  somme_{k}(d_{1}(eq0(eq(x,x_{1}))), \, \dots, \, d_p(eq0(eq(x,x_{p}))))
\end{empheq}
Avec, pour $1 \leq i \leq k$ : 
\begin{empheq}[box=\widefbox]{align*}
	& d_i(0) = y_{i} \\
        \forall n \in \mathbb{N}, \quad & d_i(n+1) = (\odot \circ p^{2}_{1})(n,d_i(n))
\end{empheq}
Et : 
\begin{empheq}[box=\widefbox]{align*}
	& somme_{1} = p^{1}_{1} \\
	 & \forall n \in \mathbb{N}^{*} \quad \forall (x_{1}, \, \dots , \, x_{n+1}) \in \mathbb{N}^{n+1},\\  
	 & somme_{n+1} : \mathbb{N}^{n+1} \to \mathbb{N} \\
	 \quad & somme_{n+1}(x_{1}, \, \dots , \, x_{n+1})  = somme(p^{n+1}_{1}(x_{1}, \, \dots , \, x_{n+1}), \, Somme_{n}(x_{1}, \dots , \, x_{n+1})) \\
\end{empheq}

\begin{empheq}[box=\widefbox]{align*}
\forall n \in \mathbb{N}^{*} \quad Somme_{n}(x_{1}, \, \dots , \, x_{n+1}) = somme_{n}(p^{n+1}_{2}(x_{1}, \, \dots , \, x_{n+1}), \, \dots, \, p^{n+1}_{n+1}(x_{1}, \, \dots , \, x_{n+1}))
\end{empheq}

\noindent Par récurrence, on obtient que :
\begin{itemize}
\item $ \forall k \in \mathbb{N}^{*},\, \, somme_{k}$ est PR 
\item $ \forall k \in \mathbb{N}-\{0,1\} \, \forall (x_{1}, \, \dots , \, x_{k}) \in \mathbb{N}^{k}, \, \, somme_{k}(x_{1}, \, \dots , \, x_{k}) = x_{1}+ \, \dots + \, x_{k}$
\end{itemize}
\ \\
De plus les $d_i$ sont PR donc $f'$ est PR.

\ \\
On constate que pour $1 \leq i \leq k$ :
$$\forall x \in \mathbb{N} \quad x = x_{i}  \Leftrightarrow d_{i}(eq0(eq(x,x_{i})))  = y_{i}$$
$$\forall x \in \mathbb{N} \quad x \neq x_{i}  \Rightarrow d_{i}(eq0(eq(x,x_{i})))  = 0$$
(eq0 agit comme un not logique ici). \newpage
\noindent Donc : 
$$ \forall x \in \mathbb{N} \quad x \notin E \implies f'(x) = 0$$
(tous les termes de la somme sont nuls)
$$ \forall 1 \leq i \leq k \quad f'(x_{i}) = y_{i}$$
(un seul terme non nul, les $x_{i}$ sont nécessairement distincts puisque $|E| = k$). \\
On en conclut $f = f'$ et $f$ est PR.
\end{solution}

\begin{problem}[Question 2]
On définit la fonction d'Ackermann via la récurrence double suivante : 
\begin{align*}
	Ack(0,x) & = x+2 \\
	Ack(1,0) & = 0 \\
	Ack(n+2,0) & = 1 \\
	Ack(n+1,x+1) & = Ack(n,Ack(n+1,x))
\end{align*}

Montrer que pour tout $n \in \mathbb{N}$, la fonction $Ack_{n} : x \mapsto Ack(n,x)$ est PR.
\end{problem}

\begin{solution}

\begin{itemize}

\item \textbf{Cas} $n = 0$ : $Ack_{0}(x) = x + 2 = s(s(x)) = s^{2}(x)$, PR par schéma de composition.
\item \textbf{Cas} $n = 1$ : 
\begin{empheq}[box=\widefbox]{align*}
 & Ack_{1}(0) = 0  \\
  \forall x \in \mathbb{N}, \quad & Ack_{1}(x+1) =  Ack(1,x+1) = Ack(0,Ack(1,x)) = (Ack_{0} \circ p^{2}_{2})(x,Ack_{1}(x))
\end{empheq}
PR par schémas de composition et de récurrence.\\

\item \textbf{Cas} $n \geq 2$ :

\begin{empheq}[box=\widefbox]{align*}
 & Ack_{n}(0) = 1  \\
  \forall x \in \mathbb{N}, \quad & Ack_{n}(x+1) =  Ack(n,x+1) = (Ack_{n-1} \circ p^{2}_{2})(x,Ack_{n}(x))
\end{empheq}

Par récurrence sur $n \geq 2$ on obtient que les $Ack_{n \geq 2}$ sont PR. \\
Ainsi, pour tout $n \in \mathbb{N}$, la fonction $Ack_{n} : x \mapsto Ack(n,x)$ est PR.
\end{itemize}

\end{solution}


\begin{problem}[Question 3]

On veut montrer dans cette question que la fonction d'Ackermann n'est pas primitive récursive.
\begin{itemize}
  \item[(\textit{a})] Montrer que $\forall n \in \mathbb{N}, \, x \in \mathbb{N}^{*}, \, Ack_{n}(x) > x$. En déduire que pour tout entier $n$, $Ack_{n}$ est strictement croissante et que $Ack$ est croissante en son premier argument : $\forall x \geq 2 \, \, \forall n \in \mathbb{N}, \, Ack(n,x) \leq Ack(n+1,x)$
    
\end{itemize}



\end{problem}

\begin{solution}

\begin{solution}[(a)]
On va procéder par induction bien fondée sur $\mathbb{N} \times \mathbb{N}^{*}$ muni de l'ordre lexicographique naturel (total et bien fondé). Il admet pour minimum : $(0,1)$. \\
On veut montrer : $$\forall n \in \mathbb{N} \, \, \forall x \in \mathbb{N}^{*} \, \, \property(n,x) : \text{"}Ack_{n}(x) > x\text{"}$$

\begin{itemize}
\item 
$\property(0,1)$ : $Ack_{0}(1) = 3 > 1$

\item \textbf{Induction} : Soit $(n,x) \in \mathbb{N} \times \mathbb{N}^{*}$ (et donc $x > 0$), on suppose $\property$ vraie pour tous les couples lexicographiquement strictement plus petits. Si $n = 0$ on conclut car $x+2 > x$, sinon :
\begin{align*}
	Ack_{n}(x) = Ack_{n}((x-1)+1) = Ack_{n-1}(Ack_{n}(x-1)) > Ack_{n}(x-1) > x-1
\end{align*}
Ceci par hypothèse d'induction car $\forall a \in \mathbb{N}^{*} (n,x) > (n-1,a)$ et $(n,x) > (n, x-1)$. \\
Par les deux inégalités strictes on en conclut : 
$$ Ack_{n}(x) > x$$

\end{itemize}

\noindent Soit $n \in \mathbb{N}$ on souhaite montrer : $$\forall x \in \mathbb{N} \, \, Ack_{n}(x+1) > Ack_{n}(x)$$
Si $n = 0$ on constate que c'est vrai. On suppose $n \geq 1$ : \\
Soit $x \in \mathbb{N}$ : 
\begin{align*}
Ack_{n}(x+1) = Ack_{n-1}(Ack_{n}(x)) > Ack_{n}(x) > x
\end{align*}

\noindent Ceci d'après la preuve précédente.\\
D'où le résultat. \\
On veut maintenant montrer que : 

$$ \forall x \geq 2 \, \, \forall n \in \mathbb{N}, \, Ack_{n}(x) \leq Ack_{n+1}(x) $$
Soit $ x \geq 2$, on peut appliquer le premier résultat à $x-1 \geq 1$, soit $n \in \mathbb{N}$ on a :
\\
$Ack_{n+1}(x-1) > x - 1$ donc 
$Ack_{n+1}(x-1) \geq x$. \\
Par croissance de $Ack_{n}$ (précédent résultat) : 

$$ Ack_{n+1}(x) = Ack_{n}(Ack_{n+1}(x-1)) \geq Ack_{n}(x)$$

\noindent D'où le résultat.


\end{solution}

\newpage

\begin{problem}
\begin{itemize}  
  \item[(\textit{b})] On pose pour $k$ entier, $Ack^{k}_{n} = Ack_{n} \circ \, \dots \, \circ Ack_{n}$, où la composition est prise $k$ fois. \\
  Montrer que $ \forall n, k, x \in \mathbb{N} \, Ack^{k}_{n}(x) \leq Ack_{n+1}(x+k)$.
  
\end{itemize}



\end{problem}



\begin{solution}[(b)]
Soient $n,x \in \mathbb{N}$ on montre par récurrence sur $k \in \mathbb{N}$  :
$$\property(k) : \text{"}Ack^{k}_{n}(x) \leq Ack_{n+1}(x+k)\text{"}$$

\begin{itemize}
\item 
$\property(0)$ : $Ack^{0}_{n}(x) = x$ on doit donc montrer :
$$ x \leq Ack_{n+1}(x)$$
Pour $x \neq 0$ on sait déjà $ x < Ack_{n+1}(x)$ (par 3a) d'où le résultat. \\
Si $x = 0$ : \begin{itemize}
			\item $Ack_{0}(0) = 2 > 0$
			\item $Ack_{1}(0) = 0 \geq 0$
			\item $Ack_{n > 1}(0) = 1 > 0$
		  \end{itemize}
		  \ \\
D'où le résultat dans tous les cas. \\


\item \textbf{Hérédité} : Soit $k \in \mathbb{N}$ on suppose $\property(k)$. On a : 

\begin{align*}
Ack^{k+1}_{n}(x) = Ack_{n}(Ack^{k}_{n}(x)) \leq Ack_{n}(Ack_{n+1}(x+k))
\end{align*}

Par croissance et hypothèse de récurrence. \\
Or $Ack_{n}(Ack_{n+1}(x+k)) = Ack_{n+1}(x+k+1)$. \\
D'où $\property(k+1)$ \\
D'où le résultat.


\end{itemize}


\end{solution}

\begin{problem}
\begin{itemize}  
  \item[(\textit{c})]  Montrer que $Ack^{k}_{n}$ est dominée par $Ack_{n+1}$.
 \end{itemize}

\end{problem}



\begin{solution}[(c)]

On procède en quatre étapes. \\ \\
\begin{problem}[Lemme 0]
\begin{itemize}
  \item[(\textit{a})] $ \forall n \geq 0 \quad Ack_{n}(1) \geq 2 $
  \item[(\textit{b})] $  \forall n \geq 0 \quad Ack_{n}(2) > 3 $
\end{itemize}
\end{problem}

\begin{solution}
\noindent \textbf{Preuve (a)} : \\
Par récurrence sur $n$ : $Ack_{0}(1) = 3 \geq 2 $ \\
$Ack_{n+1}(1) = Ack_{n}(Ack_{n+1}(0)) = Ack_{n}(1) \geq 2$ (par HR). \\
\noindent \textbf{Preuve (b)} : c'est analogue. 
\end{solution}


\newpage

\begin{problem}[Lemme 1]
Soient $k,n \in \mathbb{N}$ : \\
Soit $x_0 > 0$ tel que 
$Ack^{k}_{n}(x_{0}) \leq Ack_{n+1}(x_{0})$ \\
Alors : $ \forall x \geq x_{0} \, \, Ack^{k}_{n}(x) \leq Ack_{n+1}(x)  $
\end{problem}
\begin{solution}
\noindent \textbf{Preuve}\\
On se donne un tel $x_0$.\\
Il suffit de montrer : $ Ack^{k}_{n}(x_{0}+1) \leq Ack_{n+1}(x_{0}+1) $ \\
Si $n = 0$ et $k=0$ c'est connu par 3a. \\
On a : 

\begin{align*}
	Ack^{k}_{n}(x_{0}+1) = Ack^{k-1}_{n}(Ack_{n-1}(Ack_{n}(x_{0}))) \leq Ack^{k-1}_{n}(Ack_{n}(Ack_{n}(x_{0})))
\end{align*}
En effet, $Ack_{n}(x_{0}) \geq Ack_{n}(1) \geq 2$ par le \textbf{Lemme 0 (a)} et on conclut par croissance de $Ack$ en le premier argument (3a) et par croissance de $Ack^{k-1}_{n}$ (récurrence sur k).
Donc : \\
\begin{align*}
	Ack^{k}_{n}(x_{0}+1)  \leq Ack_{n}(Ack^{k}_{n}(x_{0})) \leq Ack_{n}(Ack_{n+1}(x_{0})) = Ack_{n+1}(x_{0}+1)
\end{align*}

D'où le résultat.

\end{solution}

\begin{problem}[Lemme 2]
Soit $n \in \mathbb{N}^{*}$ : 
$$ \forall k \in \mathbb{N} \, \, \exists x_{n,k} \in \mathbb{N}  \quad Ack_{n}(x_{n,k}) > x_{n,k} + k + 1$$


\end{problem}

\begin{solution}

Par récurrence sur $k$ : \\
\begin{itemize}
\item \textbf{$k = 0$} : par le \textbf{Lemme 0 (b)} $x_{n,0} = 2$ convient.
\item \textbf{Hérédité} ($n \neq 0$) : $Ack_{n}(x_{n,k}+1) = Ack_{n-1}(Ack_{n}(x_{n,k})) > Ack_{n-1}(x_{n,k}+k+1) $ par HR. 
On a alors : \\
\begin{align*}
Ack_{n}(x_{n,k}+1) > Ack_{n-1}(x_{n,k}+k+1) > x_{n,k}+k+1 \\
Ack_{n}(x_{n,k}+1) > x_{n,k} + k + 2
\end{align*}
$x_{n, k+1} = x_{n,k}+1$ convient donc.
\end{itemize}
\ \\
D'où le résultat.

\end{solution}

\newpage

\noindent On peut maintenant répondre à la question en démontrant : 
\begin{empheq}[box=\widefbox]{align*}
\forall n \in \mathbb{N} \, \, \forall k \in \mathbb{N} \, \, \exists C_{n,k} \in \mathbb{N} \quad \forall x \in \mathbb{N} \, \,
 Ack^{k}_{n}(x) \leq Ack_{n+1}(max(x,C_{n,k}))
\end{empheq}

\noindent Soient $n,k \in \mathbb{N}$. \\
\begin{itemize}
\item
Si $n \neq 0$, on se donne un $x_{n,k}$ du \textbf{Lemme 2}, d'après \textbf{3b} et le \textbf{Lemme 2}  : 

$$  Ack^{k}_{n}(x_{n,k}+k+1) < Ack^{k}_{n}(Ack_{n}(x_{n,k})) \leq Ack_{n+1}(x_{n,k}+k+1)$$
On pose $C_{n,k} = x_{n,k} + k + 1 > 0$
En application du \textbf{Lemme 1} à $C_{n,k}$ on a : 

$$ \forall x \geq C_{n,k} \quad Ack^{k}_{n}(x) \leq Ack_{n+1}(x)$$
\ \\
De plus, par croissance de $Ack^{k}_{n}$ :  $$Ack_{n+1}(C_{n,k}) \geq Ack^{k}_{n}(C_{n,k}) \implies \forall x \leq C_{n,k} 
\, \, Ack_{n+1}(C_{n,k}) \geq Ack^{k}_{n}(x)
$$
C'est ce qu'on voulait. \\
\item Si $n = 0$ on ne peut pas se servir du $\textbf{Lemme 2}$ mais on a par récurrence : 
$$ Ack^{k}_{0}(x) = x+2k$$
\noindent Alors par récurrence sur $k$, on montre l'existence d'un $x_{0,k} \neq 0$ tel que :
$$x_{0,k} + 2k \leq Ack_{1}(x_{0,k})$$
Pour $k=0$ tout $x \neq 0$ convient par (3a). \\
L'argument d'hérédité est le même que pour le lemme 2, $x_{0,k+1}=x_{0,k}+1$ convient. \\
Muni de cela on peut conclure comme dans le cas $n \neq 0$ en prenant $C_{0,k}=x_{0,k}$. 
\end{itemize}
\ \\
D'où le résultat. (ouf!!)
\end{solution}

\begin{problem}
\begin{itemize}  
  \item[(\textit{d})]  Montrer que si $f$ PR est définie avec $n$ utilisations du 
  schéma de récurrence , alors $Ack_{n+1}$ domine $f$.
 \end{itemize}
 \end{problem}
 
 \ \\
 \begin{solution}[(d)]
\noindent On procède par récurrence sur $n$. En montrant la propriété : 

$$ 
\property(n) : \text{"} \forall f \text{  PR  } \text{f construite avec } n \text{ utilisations du SR} \implies
\exists k \in \mathbb{N} \, \, Ack^{k}_{n+1} \text{ domine } f \text{"}
$$
\newpage

\begin{itemize}

\item $n = 0$ : on vérifie que $\odot$, $s$ et les $p_{i}^{k}$ sont dominées par $Ack^{1}_{1}$ (on utilise 3a). \\
Ensuite, par distinction de multiples cas on constate que $f$ définie par schéma de composition avec $h$ et $g_{1}
 \, \dots \,g_{p}$ des fonctions de base est dominée par $Ack_{1}$.
 
 \item \textbf{Hérédité} : On suppose la propriété vraie pour les fonctions définies à l'aide de $k \leq n$ utilisations du schéma de récurrence.
Soit $f$ définie à l'aide de $n+1$ utilisations, on peut supposer sans perte de généralité que : 
\begin{align*}
f(a_{1}, \, \dots , \, a_{p}, 0) & = g(a_{1}, \, \dots , \, a_{p}) \\
f(a_{1}, \, \dots , \, a_{p}, x+1) & = h(a_{1}, \, \dots , \, a_{p}, \, x , \,  f(a_{1}, \, \dots , \, a_{p}, x))
\end{align*}
Avec $h$ et $g$ PR définie avec au plus $n$ utilisations du schémas de récurrence.
En effet car si $f$ est en étape finale contruite par schéma de composition on ramène l'étude aux fonctions qui interviennent. \\

Par HR on a l'existence de $k_{1}$, $C_{1}$ et $k_{2}$, $C_{2}$ tels que :

\begin{align*}
\forall a_{1}, \, \dots , \, a_{p} \quad g(a_{1}, \, \dots , \, a_{p}) & \leq  Ack^{k_{1}}_{n+1}(sup(a_{1}, \, \dots , \, a_{p}, C_{1}))\\
\forall a_{1}, \, \dots , \, a_p{+1}, \, a_{p+2} \quad  h(a_{1}, \, \dots , \, a_{p+1}, \, a_{p+2}) & \leq  Ack^{k_{2}}_{n+1}(sup(a_{1}, \, \dots , \, a_{p+1}, \, a_{p+2}, C_{2}))\\
\end{align*}

En effet si $g$ ou $h$ sont construits avec moins de $n$ applications du SR, la croissance de $Ack$ en le premier argument permet quand même d'établir l'inégalité. \\
Par récurrence sur $x$ on montre qu'on a alors :
$$ f(a_{1}, \, \dots , \, a_{p}, x) \leq Ack^{k_{1} + xk_{2}}_{n+1}(sup(a_{1}, \, \dots , \, a_{p}, \, x, \,  C_{1}, C_{2}))  $$

	\begin{itemize}[leftmargin=*]
		\item x = 0 : $C_{2} \leq C_{1}$, c'est connu. $C_{2} > C_{1}$ par croissance.
		\item \textbf{Hérédité} :
		
		\begin{align*}
			f(a_{1}, \, \dots , \, a_{p}, x+1) & \leq Ack^{k_{2}}_{n+1}(sup(a_{1}, \, \dots , \, a_{p}, \, x \, , f(a_{1}, \, \dots , \, a_{p}, x) )) \\
			& \leq Ack^{k_{2}}_{n+1}(sup(a_{1}, \, \dots , \, a_{p}, \, x \, , Ack^{k_{1} + xk_{2}}_{n+1}(sup(a_{1}, \, \dots , \, a_{p},  \, x, \, C_{1}, C_{2})) ))
		\end{align*}
		\noindent Or : 
		\begin{align*}
		Ack^{k_{1} + xk_{2}}_{n+1}(sup(a_{1}, \, \dots , \, a_{p},  \, x, \, C_{1}, C_{2})) & \geq  Ack^{k_{1} + xk_{2}}_{n+1}(a_{1}) \geq a_{1} \\
		& \vdots  \\
		Ack^{k_{1} + xk_{2}}_{n+1}(sup(a_{1}, \, \dots , \, a_{p},  \, x, \, C_{1}, C_{2})) & \geq  Ack^{k_{1} + xk_{2}}_{n+1}(x) \geq x
		\end{align*}
		Donc :
		\begin{align*}
			f(a_{1}, \, \dots , \, a_{p}, x+1) 
& \leq Ack^{k_{2}}_{n+1}(Ack^{k_{1} + xk_{2}}_{n+1}(sup(a_{1}, \, \dots , \, a_{p},  \, x, \, C_{1}, C_{2})) ) \\
		& \leq Ack^{k_{1} + (x+1)k_{2}}_{n+1}(sup(a_{1}, \, \dots , \, a_{p},  \, x, \, C_{1}, C_{2})) 
		\end{align*}
		D'où le résultat.
	\end{itemize}

\newpage 

\noindent D'où : 
\begin{align*}
f(a_{1}, \, \dots , \, a_{p}, x) & \leq  Ack^{k_{1} + xk_{2}}_{n+1}(sup(a_{1}, \, \dots , \, a_{p}, \, x, \,  C_{1}, C_{2})) \\
& \leq  Ack_{n+2}(sup(a_{1}, \, \dots , \, a_{p}, \, x, \,  C_{1}, C_{2})+k_{1} + xk_{2})
\end{align*}

TODO

\end{itemize}

\end{solution}

\begin{problem}
\begin{itemize}  
  \item[(\textit{e})]  Conclure. \end{itemize}
 \end{problem}
 
  \begin{solution}[(e)]
  Supposons par l'absurde que $Ack$ est PR. Alors : $f: \, \, n \mapsto Ack(n,n+1)$ est PR par composition. Supposons $f$ construite avec $m$ utilisations du SR.\\
   On a montré précédemment qu'alors il existe $C$ tel que $\forall n > C$ :
   
   $$ Ack(n,n+1) \leq Ack_{m+1}(n) $$
   \ \\
   
  \noindent Maintenant, si $n > m+1$ par 3a on a :
   
   $$ Ack_{m+1}(n) < Ack_{m+1}(n+1) \leq Ack_{n}(n+1) = Ack(n,n+1) $$ 
   \ \\
   \noindent Absurde en prenant $n > sup(n,C)$.
   \\
   D'où le résultat.
   \end{solution}

\end{solution}

\begin{problem}[Question 4]

On définit l'ensemble $R$ des fonctions récursives comme le plus petit ensemble de fonctions, éventuellement partielle, qui contient PR et qui est clos par le \textit{schéma de minimisation non-borné} : si $f : A \subseteq \mathbb{N}^{p+1} \to \mathbb{N}$ est dans $R$, alors la fonction $g : B \subseteq \mathbb{N}^{p} \to \mathbb{N}$ est dans $R$, où $g$ est définie par :

$$ g(a_{1}, \, \dots \,, a_{p}) = min \{ z : f(a_{1}, \, \dots , \, a_{p} , \, z) = 0 \} \text{ si non vide}$$
 
Quelle est la cardinalité de $R$ ? Exhiber une fonction qui n'est pas dans $R$.

\end{problem}

\begin{solution}
Si on pose : 
\begin{align*}
& R_{0}  = \{ \odot, \, p^{k}_{i}, \, s \} \\
& R_{n+1}  = \\ 
& R_{n} \cup ( \cup_{p} \cup_{m} \{ SC(h,g_{1}, \, \dots \, , g_{p}) \, \, | \, \, h \in  N_{p} \cap R_{n} \, \, g_{i \leq p} \in N_{m} \cap R_{n}  \} ) \\ 
& \cup (\cup_{p} \{ SR(g,h) \, \, | \, \, g \in N_{p} \cap R_{n} \, \, h \in N_{p+2} \cap R_{n} \}) \\
& \cup (\cup_{p} \{ SMNB(f) \, \, | \, \, f \in N_{p+1} \cap R_{n} \})
 \end{align*}
 
 \newpage
 
 Avec $N_{p}$ les fonctions (pouvant être partielles) de $\mathbb{N}^{p} \to \mathbb{N}$. Et $SC$, $SR$, $SMNB$ les différents schémas introduits. Alors d'après la théorie des ensembles inductifs (PROG) on sait que :
 
 $$ R = \cup_{n} R_{n} $$

\noindent Par récurrence on a que les $R_{n}$ sont dénombrables par réunion dénombrable d'ensembles dénombrables ($N_{p}\cap R_{n}$ dénombrable). \\
Ainsi $R$ est réunion dénombrable d'ensembles dénombrables : $R$ est dénombrable. \\
\\
Par argument de cardinalité on a donc l'existence de fonctions non R. On peut en exhiber une par argument diagonal. \\
On pose $R = \{ f_{n} \}$ et : 

$$ g(n) = f_{n}(n) + 1$$

\noindent Supposons par l'absurde que $g$ est R. Il existe $m$ tel que $g = f_m$ et alors : 
$$ g(m) = f_{m}(m) + 1 \neq f_{m}(m)$$

\noindent Absurde.

\end{solution}

\begin{problem}[Question 5]
Montrer que si $f$ est R, alors $f$ est calculable par une machine de Turing.
\end{problem}

\begin{solution}
On raisonne par induction structurelle (on prend des entrées en binaire, machine à deux rubans I/O) :
\begin{itemize}
\item \textbf{Cas} $\odot$ : une machine qui écrit 0 quelque soit l'entrée.
\item \textbf{Cas} $p^{k}_{i}$ : on sépare les différents paramètre d'un espace sur le ruban, il suffit de maintenir un compteur pour atteindre $i$ et recopier l'entrée sur la sortie.
\item \textbf{Cas} $s$ : on sait ajouter $+1$ avec une MT en binaire (cf TD).

\item \textbf{Cas SC} : soit $h \in N_{p} \cap R$ et $g_{1}, \, \dots , \, g_{p} \in N_{n} \cap R$. On   
considère $f = SC(h,g_{1}, \, \dots , \, g_{p})$. \\
Par hypothèse d'induction, on dispose de $M_{0}$ qui calcule $h$ et $M_{1} , \, \dots , \, M_{p}$ qui simulent  $g_{1}, \, \dots , \, g_{p}$. Il suffit de composer ces machines en faisant écrire les résultats de $M_{1} , \, \dots \, , M_{p}$ (sur $x_{1} , \, \dots \, x_{n}$) séparés d'un espace sur le ruban d'entrée de $M_{0}$ et appliquer $M_{0}$.

\item \textbf{Cas SR} : Soit $f = SR(g,h)$, $M_{0} = M(f)$ et $M_{1} = M(h)$. Pour calculer $f$ il faut simuler la stack comme sur un ordi, on empile sur le ruban les arguments successifs jusqu'à $x = 0$ auquel cas on calcul avec $M_{0}$ et après on remonte tant que le ruban n'est pas vide en appliquant $M_{1}$ et en utilisant le résultat comme paramètre manquant de l'appel d'après.

\item \textbf{Cas SMNB} : Soit $g = SMB(f)$ et $M_{0} = M(f)$. On interprète la non définition d'une fonction en un point comme la non terminaison sur cette entrée de la machine qui calcule $f$. Muni de cette définition on construit une machine qui teste tous les $z$ dans l'ordre croissant en utilisant le calcul de $M_{0}$. Si un tel $z$ existe on l'écrit sur le ruban sinon le calcul ne termine pas : $g$ n'est pas définie.
\end{itemize}


\end{solution}


\end{document}
